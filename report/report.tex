%------------------------------------------------------------------------------
%  report.tex
%------------------------------------------------------------------------------
%
%	BA6 - Database systems
%
%	Authors :
%		203267 - Bastien Antoine
%		183785 - Denoréaz Thomas
%		185078 - Dieulivol David
%
%	Versions :
%		2013.03.30 - Initial version
%

%--------------DOCUMENT--------------------------------------------------------
\documentclass[a4paper,oneside,11pt]{report}  % Type de document
\usepackage[english]{babel}

\usepackage{amsmath}
\usepackage{amsfonts}
\usepackage{amssymb}
\usepackage{mathspec}
\usepackage{fontspec}
\usepackage{xltxtra}
\usepackage{xunicode}
\usepackage[table]{xcolor}

%--------------PACKAGES--------------------------------------------------------
\usepackage[Bjornstrup]{fncychap}           % Chapitres
\usepackage{indentfirst}

\usepackage{fancyhdr}                       % Entete et pied de pages
\usepackage[outerbars]{changebar}           % Positionnement barre en marge externe
\usepackage[table]{xcolor}
\usepackage{lastpage}
%\usepackage{subfigure}
\usepackage{natbib}
\usepackage{paralist}
\usepackage{makeidx}                        % Permet de créer une indexation
\usepackage{multicol}                       % Gestion de plusieurs colonnes
\usepackage{multirow}
\usepackage{array}                          % 
\usepackage{a4wide}                         % Utilisation de toute la page A4
\usepackage[hmargin=2.5cm,vmargin=2.5cm]{geometry}

\usepackage{listings}                       % Affichage de code source

\usepackage{pifont}                         % Polices supplementaires
\usepackage{textcomp}
\usepackage{pgf}
\usepackage{tikz}
\usepackage{ae}                             % Affichage sous Adobe Reader
\usepackage{ifpdf}
\usepackage{url}
\usepackage{wrapfig}

\usepackage{graphicx}
\usepackage{caption}
\usepackage{subcaption}

\usepackage[disable]{todonotes}
%\usepackage{todonotes}

\usepackage{bytefield}

\usepackage{pgf}
\usepackage{tikz}
\usetikzlibrary{shapes}
\usetikzlibrary{arrows}
\usetikzlibrary{automata}

\DeclareTextCommandDefault{\nobreakspace}{\leavevmode\nobreak\ } 

%--------------PERSONNAL-INFORMATIONS------------------------------------------
\newcommand{\theversion}{3.0}

\newcommand{\myself}{Bastien Antoine, Denoréaz Thomas, Dieulivol David}
\newcommand{\titre}{Olympic Games}
\newcommand{\doctype}{Introduction to Database Systems}

%--------------MACROS----------------------------------------------------------
\newcommand{\nl}{\newline\indent}
\newcommand{\nol}{\newline\noindent}
\newcommand{\mathbox}[1]{\;\mbox{#1}\;}

%--- Elementary Algebra ---
\newcommand{\N}{\mathbb{N}}
\newcommand{\Nn}{\mathbb{N}^{*}}
\newcommand{\Z}{\mathbb{Z}}
\newcommand{\Zn}{\mathbb{Z}^{*}}
\newcommand{\Q}{\mathbb{Q}}
\newcommand{\R}{\mathbb{R}}
\newcommand{\Rp}{\mathbb{R}_{+}}
\newcommand{\Rm}{\mathbb{R}_{-}}
\newcommand{\Rn}{\mathbb{R}^{*}}
\newcommand{\Rpn}{\mathbb{R}_{+}^{*}}
\newcommand{\Rmn}{\mathbb{R}_{-}^{*}}
\newcommand{\C}{\mathbb{C}}

%\newcommand{\M}{\mathbb{M}}

\newcommand{\FF}{\mathcal{F}}
\newcommand{\GG}{\mathcal{G}}
\newcommand{\RR}{\mathcal{R}}
\newcommand{\PP}{\mathcal{P}}

\newcommand{\imp}{\Rightarrow}
\newcommand{\pmi}{\Leftarrow}
\newcommand{\equ}{\Leftrightarrow}
\newcommand{\Imp}{\;\Rightarrow\;}
\newcommand{\Pmi}{\;\Leftarrow\;}
\newcommand{\Equ}{\;\Leftrightarrow\;}

\newcommand{\nsubset}{\not\subset}
\newcommand{\ot}{\rightarrow}

\newcommand{\eqnpar}[1]{\left\{\begin{array}{lcl}#1\end{array}\right.}
\newcommand{\pareqn}[1]{\left.\begin{array}{lcl}#1\end{array}\right\}}

\newcommand{\appl}[5]{\begin{array}{llll}#1:&#2&\to&#3\\&#4&\mapsto&#5\end{array}}

\newcommand{\ddef}{D_{def}}

\newcommand{\liff}{\leftrightarrow}

%--- Geometry ---
\newcommand{\vect}[1]{\overrightarrow{#1}}
\newcommand{\coord}[1]{\left(\begin{array}{ccc}#1\end{array}\right)}

%--- Relational Algebra ---
\newcommand{\select}[2]{\left(\sigma_{#1}\ \entity{#2}\right)}
\newcommand{\project}[2]{\pi_{#1}\ \left(#2\right)}
\newcommand{\rename}[2]{\rho\left(\entity{#1},\ #2\right)}
\newcommand{\renamep}[3]{\rho\left(\entity{#1}\left(#2\right),\ #3\right)}
\newcommand{\join}{\;\bowtie\;}
\newcommand{\entity}[1]{\mbox{#1}}

% ----- LISTINGS ------------------------------------------------------------
\definecolor{ltgray}{rgb}{0.98,0.98,0.98}
\definecolor{ltbluegray}{rgb}{0.97,0.97,1}
\definecolor{dkred}{rgb}{.4,0,0}
\definecolor{ltred}{rgb}{.75,0,0}
\definecolor{dkgreen}{rgb}{0,0.5,0}
\definecolor{ltgreen}{rgb}{0,0.75,0}
\definecolor{dkblue}{rgb}{0,0,.4}
\definecolor{ltblue}{rgb}{0,0,.75}
\definecolor{dkcyan}{rgb}{0,0.4,0.4}
\definecolor{ltcyan}{rgb}{0,0.75,0.75}

\definecolor{nameclr}{rgb}{0,0.4,0.6}
\definecolor{ltnameclr}{rgb}{0.7,0.9,1}
\definecolor{dknameclr}{rgb}{0,0.2,0.3}
\newcommand{\namesty}[1]{{\color{nameclr}\textit{#1}}}

\lstset{
  language          = SQL,                  % langage par dfaut
  captionpos        = b,                    % position du caption
  numbers           = left,                 % affichage des numro de lignes
  frame             = trBL,                 % cadre du listing
  tabsize           = 3,                    % taille de la tabulation
  showtabs          = false,                % type de tabulation
  showspaces        = false,                % type d'espace
  showstringspaces  = false,                % type d'espace pour les chanes
  basicstyle        = \small\ttfamily,      % style du listing
  commentstyle      = \color{dkgreen}\itshape, % Comment Style
  stringstyle       = \color{ltred}\slshape, % Strings Style
  keywordstyle      = \color{nameclr}\textbf, % Keyword style
  breaklines        = true,                 % permet de sparer les lignes
  breakatwhitespace = true,                 % spare strictement aux espaces
  postbreak         = \Pisymbol{pzd}{229},  % symbol devant la coupure
  backgroundcolor   = \color{ltbluegray},
  linewidth         = \textwidth,
  xleftmargin       = 0.05\textwidth,
  xrightmargin      = 0.05\textwidth
}

% --- Macro pour les listings ---
\newcommand{\inputlisting}[1]
{\lstinputlisting[caption=#1]{dev/#1}}

% ----- PDF -----------------------------------------------------------------
\usepackage{graphicx, color}        % insertion images et couleurs
\graphicspath{{pic/}}
\DeclareGraphicsExtensions{.jpg,.png,.JPG}  % Formats d'images
\usepackage{pslatex}                        % Polices PDF, moins lourdes et non bitmap

\usepackage[                         %   Paramètrage de la navigation
    bookmarks         = true,               % Signets
    bookmarksnumbered = true,               % Signets numérotés
    pdfpagemode       = UseNone,            % Signets/vignettes fermé à l'ouverture
    pdfstartview      = FitR,               % La page prend toute la largeur
    pdfpagelayout     = OneColumn,          % Vue par page
    colorlinks        = false,              % Liens en couleur
    pdfborder         = {0 0 0}             % Style de bordure
    ]{hyperref}                             % Utilisation de HyperTeX

\usepackage{hypcap}
    
\hypersetup{                                % Information sur le document
    pdfauthor   = {\myself},
    pdftitle    = {\titre},
    pdfsubject  = {\doctype},
    plainpages  = false
}

\usepackage{pdfpages}                       % permet d'inclure des fichiers entiers pdf

%--------------ENTETE-ET-PIED-DE-PAGE------------------------------------------
\pagestyle{headings}

\pagestyle{fancy}
\renewcommand{\headrulewidth}{0.5pt}
\renewcommand{\footrulewidth}{0.5pt}

\lhead{}
\chead{}
%\rhead{}
\lfoot{\textcopyright EPFL - IC - Version \theversion}
\cfoot{}
\rfoot{\thepage\ on \pageref{LastPage}}

%--------------PAGE-DE-GARDE---------------------------------------------------

\title{{\huge \doctype}\\\vspace{1em}{\Huge \titre}\\\vspace{4em}\includegraphics[width=0.5\textwidth]{logo_epfl}}
\author{Bastien Antoine (203267)\\ Denoréaz Thomas (183785)\\ Dieulivol David (185078)}
\date{Academic years 2012-2013\\\texttt{(\today)}}

\makeindex

\newcommand{\insertblankpage}{\newpage\thispagestyle{empty}\mbox{}\newpage}
\newcommand{\compref}[1]{\ref{#1} on page~\pageref{#1}}
\newcommand{\todoline}[1]{\todo[inline]{\textbf{TODO:} #1}}
\newcommand{\xxxline}[1]{\todo[inline]{\textbf{XXX:} #1}}
\newcommand{\fixmeline}[1]{\todo[inline]{\textbf{FIXME:} #1}}

\newcommand{\bitsbox}[3][lrt]{\bitbox[#1]{#2}{\raisebox{-3em}{#3}}}


%\newcommand{\descbox}[2]{\parbox[c][3.8\baselineskip]{0.95\width}{%
% facilitates the creation of memory maps. Start address at the bottom,
% end address at the top.
% syntax:
%   \memsection{end address}{start address}{height in lines}{text in box}
\newcommand{\memsection}[4]{%
  % define the height of the memsection
  \bytefieldsetup{bitheight=#3\baselineskip}%
  \bitbox[]{10}{%
    \texttt{#1}%      print end address
    \\
    %   do some spacing
    \vspace{#3\baselineskip}
    \vspace{-2\baselineskip}
    \vspace{-#3pt}
    \texttt{#2}%
  }%
  \bitbox{16}{#4}% 
}



% ----- DEBUT DU DOCUMENT ---------------------------------------------------
\begin{document}

% ----- Title + Logos -----

% --- Logo LAP + EPFL ---
%\begin{figure}[t]
%	\centering
%	\includegraphics[width=\textwidth]{header}
%	\subfigure{
%		\includegraphics[height=6em]{logo_lap}
%		\hspace*{\fill}
%		\raisebox{2.65em}{
%			\begin{tabular}{c}
%				\textsf{\textbf{\Large Processor Architecture Laboratory}}\\
%				\textsf{\textbf{\Large Laboratoire d'Architecture des Processeurs}}\\
%				\\
%				\textsf{School of Computer and Communication Sciences}
%			\end{tabular}
%		}
%		\hspace*{\fill}	
%		\includegraphics[height=6em]{logo_epfl}
%	}
%\end{figure}

\begin{figure}[b]
	\centering
	\includegraphics[height=6em]{logo_mysql}
	\hspace*{\fill}
	\includegraphics[height=3em]{logo_play}
	\hspace*{\fill}	
	\includegraphics[height=3em]{logo_scala}
\end{figure}

\newpage

\maketitle                                  % Titre du document

% ----- Quote -----

%\insertblankpage
%\newpage

%\thispagestyle{empty}
%\vspace*{\fill}

%\begin{center}
%\parbox{0.73\textwidth}{
%\large \emph{\textquotedblleft The most important motivation for the research work that resulted in the relational model was the %objective of providing a sharp and clear boundary between the logical and physical aspects of database %management.\textquotedblright }}
%\parbox{0.65\textwidth}{\begin{flushright}---\hspace{0.5em} Edgar F. Codd, February 1982\end{flushright}}
%\end{center}

%\vspace*{\fill}

% ----- Contents + Preface -----
\pagenumbering{roman}                       % Numerotation romaine

\tableofcontents                            % Table des matieres
%\listoffigures                              % Liste des images
%\listoftables                               % Liste des tableaux
%\lstlistoflistings                          % Liste des listings
\newpage

% ----- Main Document -----
\pagenumbering{arabic}                      % Numerotation arabe
%\input{tex/technologies}
%------------------------------------------------------------------------------
%  er-model.tex
%------------------------------------------------------------------------------
%
%	BA6 - Database systems
%
%	Authors :
%		203267 - Bastien Antoine
%		183785 - Denoréaz Thomas
%		185078 - Dieulivol David
%
%	Versions :
%		2013.03.30 - Initial version
%

\chapter{Entity–relationship model}

\begin{center}
	\fbox{\includegraphics[width=0.6\textwidth]{er-model}}
\end{center}

From the analysis of the Dataset, here are our assumptions:

\begin{itemize}
	\item[$\circ$] An \textbf{Athlete} is always performing a \textbf{Discipline} instead of just a \textbf{Sport}.
	\item[$\circ$] An \textbf{Athlete} can represent only a \textbf{Country} for a \textbf{Game}. However, he can represent another \textbf{Country} for another \textbf{Game}.
	\item[$\circ$] A \textbf{Game} can only be hosted by one and only one \textbf{Country}, but this \textbf{Country} can host several \textbf{Games}.
	\item[$\circ$] Each \textbf{Discipline} is defined by its \textbf{Sport}.
	\item[$\circ$] An \textit{Event} is characterized by only a \textbf{Game} and only a \textbf{Discipline}.
	\item[$\circ$] A \textit{Medal} is obtained for a \textit{Representant} during an \textit{Event}.
	\item[$\circ$] A \textit{Participant} is formed by both a \textit{Representant} and an \textit{Event}.
\end{itemize}



%------------------------------------------------------------------------------
% END OF DOCUMENT
%------------------------------------------------------------------------------
%------------------------------------------------------------------------------
%  ddl-statements.tex
%------------------------------------------------------------------------------
%
%	BA6 - Database systems
%
%	Authors :
%		203267 - Bastien Antoine
%		183785 - Denoréaz Thomas
%		185078 - Dieulivol David
%
%	Versions :
%		2013.03.30 - Initial version
%

\chapter{Relational schema and constraints}

\section{Relational schema}

\begin{figure}[h!]
	\centering
	\fbox{\includegraphics[width=0.9\textwidth]{ddl-scheme-new}}
	\caption{Generated EER Model from MySQL Workbench.\label{fig:ddl-scheme}}
\end{figure}
After implementing the DDL from Section \ref{sec:ddl}, we generated the scheme in Figure \ref{fig:ddl-scheme} using MySQL Workbench.

\newpage
\section{SQL Data definition language statements}
\label{sec:ddl}

We decided to implement our project, using the Oracle MySQL database management system. Following is the listing of our entities and relations.

We changed the DDL adding the unique constraints because we had to much duplicate during the import.

\begin{center}
	\lstinputlisting[caption={DDL Entities}]{scripts/ddl-entities.sql}
\end{center}
\newpage

\begin{center}
	\lstinputlisting[caption={DDL Relations}]{scripts/ddl-relations.sql}
\end{center}


%------------------------------------------------------------------------------
% END OF DOCUMENT
%------------------------------------------------------------------------------
%------------------------------------------------------------------------------
%  data-import.tex
%------------------------------------------------------------------------------
%
%	BA6 - Database systems
%
%	Authors :
%		203267 - Bastien Antoine
%		183785 - Denoréaz Thomas
%		185078 - Dieulivol David
%
%	Versions :
%		2013.03.30 - Initial version
%

\chapter{Data importation}

The main issue that appeared while importing the data is that we cannot have the discipline for each athlete. This is problematical for the relation \textit{Representant\_participates\_Event}. There is information in the csv files about the sport that practices each athlete but the sport cannot define an event. So we have decided not to set the discipline as a primary key so that all non- medalists can still be stored in the database.

\begin{center}
	\lstinputlisting[language=Ruby, caption={Ruby importation script}]{scripts/import_script.rb}
\end{center}



%------------------------------------------------------------------------------
% END OF DOCUMENT
%------------------------------------------------------------------------------
%------------------------------------------------------------------------------
%  queries.tex
%------------------------------------------------------------------------------
%
%	BA6 - Database systems
%
%	Authors :
%		203267 - Bastien Antoine
%		183785 - Denoréaz Thomas
%		185078 - Dieulivol David
%
%	Versions :
%		2013.03.30 - Initial version
%

\chapter{Queries}

Here are some explanations of queries that seem difficult to understand:
\begin{itemize}
	\item[$\circ$] The query A is the intersection of athletes who won medals in summer and who won in
	winter.
	\item[$\circ$] The query C is selecting the minimum year (so the first event) where a country won its
first medal. It returns for each country the corresponding Olympics which mean the host
city and the year.
	\item[$\circ$] The query D is selecting the union of the best country (most of medals) of all of the winter
Olympics and the best one of all of the summer Olympics.
	\item[$\circ$] The query G is taking for each Olympics the maximum of all counts of participants in
each country.
\end{itemize}

\begin{center}
	\lstinputlisting[caption={Query A}]{scripts/A.sql}
\end{center}

\begin{center}
	\lstinputlisting[caption={Query B}]{scripts/B.sql}
\end{center}

\begin{center}
	\lstinputlisting[caption={Query C}]{scripts/C.sql}
\end{center}

\begin{center}
	\lstinputlisting[caption={Query D}]{scripts/D.sql}
\end{center}

\begin{center}
	\lstinputlisting[caption={Query E}]{scripts/E.sql}
\end{center}

\begin{center}
	\lstinputlisting[caption={Query F}]{scripts/F.sql}
\end{center}

\begin{center}
	\lstinputlisting[caption={Query G}]{scripts/G.sql}
\end{center}

\begin{center}
	\lstinputlisting[caption={Query H}]{scripts/H.sql}
\end{center}


%------------------------------------------------------------------------------
% END OF DOCUMENT
%------------------------------------------------------------------------------
%------------------------------------------------------------------------------
%  web.tex
%------------------------------------------------------------------------------
%
%	BA6 - Database systems
%
%	Authors :
%		203267 - Bastien Antoine
%		183785 - Denoréaz Thomas
%		185078 - Dieulivol David
%
%	Versions :
%		2013.04.21 - Initial version
%

\chapter{Web}

\section{Entities \& relations}

Here is a view that show the listing of an entity or a relation, we can see that the SQL statement is above the table.
We can directly edit or remove an entry when clicking on the edit or delete link.

\begin{figure}[!ht]
	\centering
	\includegraphics[width=0.9\textwidth]{web3_Entities}
	\caption{Entities listing\label{fig:ddl-scheme}}
\end{figure}

\newpage
\section{Query view}

The result of each query is shown inside a table, a description and the SQL statement are above the results.

\begin{figure}[!ht]
	\centering
	\includegraphics[width=0.9\textwidth]{web3_QueryM}
	\caption{Query view\label{fig:ddl-scheme}}
\end{figure}

\newpage
\section{Add entity / relation}

To change data inside the databases, we can add, edit and remove entities and relations using the WEB UI.

\begin{figure}[!ht]
	\centering
	\includegraphics[width=0.9\textwidth]{web3_add}
	\caption{Add entity form\label{fig:ddl-scheme}}
\end{figure}


\section{Custom Query}

To perform custom query, we implement a small interface to insert SQL Code and then query the DB and print the result.
If an error occurs a message is displayed.

\begin{figure}[!ht]
	\centering
	\includegraphics[width=0.9\textwidth]{web3_add}
	\caption{Add entity form\label{fig:ddl-scheme}}
\end{figure}



%------------------------------------------------------------------------------
% END OF DOCUMENT
%------------------------------------------------------------------------------

% ----- Appendices -----
%\appendix
%\part*{Appendix}
%\addcontentsline{toc}{part}{Appendix}

%\input{tex/appendixA}

% ----- References -----
%\newpage
%\normalsize
%\nocite{*}
%\renewcommand{\bibname}{References}
%\bibliographystyle{unsrtnat}
%\begin{small}
%\bibliography{bib/bibliographie}
%\end{small}

% ----- Index -----
%\newpage
%\addcontentsline{toc}{chapter}{Index}
%\printindex

% ----- Index -----
\newpage
\addcontentsline{toc}{chapter}{Todo}
\listoftodos

\end{document}

% ----------------------------------------------------------------------------
% ----- END OF DOCUMENT ------------------------------------------------------
% ----------------------------------------------------------------------------