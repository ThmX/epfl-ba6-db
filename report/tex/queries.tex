%------------------------------------------------------------------------------
%  queries.tex
%------------------------------------------------------------------------------
%
%	BA6 - Database systems
%
%	Authors :
%		203267 - Bastien Antoine
%		183785 - Denoréaz Thomas
%		185078 - Dieulivol David
%
%	Versions :
%		2013.03.30 - Initial version
%

\chapter{Queries}

Here are some explanations of queries that seem difficult to understand:
\begin{itemize}
	\item[$\circ$] The query A is the intersection of athletes who won medals in summer and who won in
	winter.
	\item[$\circ$] The query C is selecting the minimum year (so the first event) where a country won its
first medal. It returns for each country the corresponding Olympics which mean the host
city and the year.
	\item[$\circ$] The query D is selecting the union of the best country (most of medals) of all of the winter
Olympics and the best one of all of the summer Olympics.
	\item[$\circ$] The query G is taking for each Olympics the maximum of all counts of participants in
each country.
\end{itemize}

\begin{center}
	\lstinputlisting[caption={Query A}]{scripts/A.sql}
\end{center}

\begin{center}
	\lstinputlisting[caption={Query B}]{scripts/B.sql}
\end{center}

\begin{center}
	\lstinputlisting[caption={Query C}]{scripts/C.sql}
\end{center}

\begin{center}
	\lstinputlisting[caption={Query D}]{scripts/D.sql}
\end{center}

\begin{center}
	\lstinputlisting[caption={Query E}]{scripts/E.sql}
\end{center}

\begin{center}
	\lstinputlisting[caption={Query F}]{scripts/F.sql}
\end{center}

\begin{center}
	\lstinputlisting[caption={Query G}]{scripts/G.sql}
\end{center}

\begin{center}
	\lstinputlisting[caption={Query H}]{scripts/H.sql}
\end{center}


%------------------------------------------------------------------------------
% END OF DOCUMENT
%------------------------------------------------------------------------------